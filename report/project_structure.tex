%%%%%%%%%%%%%%%%%%%%%%%%%%%%%%%%%%%%%%%%%%%%%%%%%%%%%%%%%%%
% 项目结构文档 - 流形学习方法比较
%%%%%%%%%%%%%%%%%%%%%%%%%%%%%%%%%%%%%%%%%%%%%%%%%%%%%%%%%%%
\documentclass{article}

% 基本包
\usepackage[UTF8]{ctex} % 中文支持
\usepackage[margin=2.5cm]{geometry}
\usepackage{hyperref}
\usepackage{graphicx}
\usepackage{listings}
\usepackage{xcolor}
\usepackage{amsmath}
\usepackage{booktabs}

% 代码样式设置
\definecolor{codegreen}{rgb}{0,0.6,0}
\definecolor{codegray}{rgb}{0.5,0.5,0.5}
\definecolor{codepurple}{rgb}{0.58,0,0.82}
\definecolor{backcolour}{rgb}{0.95,0.95,0.95}

\lstdefinestyle{mystyle}{
    backgroundcolor=\color{backcolour},   
    commentstyle=\color{codegreen},
    keywordstyle=\color{codepurple},
    stringstyle=\color{codegreen},
    basicstyle=\ttfamily\small,
    breakatwhitespace=false,         
    breaklines=true,                 
    captionpos=b,                    
    keepspaces=true,                 
    numbersep=5pt,                  
    showspaces=false,                
    showstringspaces=false,
    showtabs=false,                  
    tabsize=2
}

\lstset{style=mystyle}

\title{流形学习方法比较项目\\结构与代码说明}
\author{丁丁}
\date{\today}

\begin{document}

\maketitle

\section{项目概述}

本项目比较了九种降维方法在人脸姿态排序任务上的性能。具体来说,我们使用了一个包含33张同一个人不同视角的灰度图像的数据集,尝试恢复图像的1维旋转流形,并根据重建的流形对人脸姿态进行排序。

项目中实现和评估了以下降维方法:
\begin{itemize}
    \item 线性方法:主成分分析 (PCA)、多维缩放 (MDS)
    \item 非线性流形方法:Isomap、局部线性嵌入 (LLE)、局部切空间对齐 (LTSA)、改进的LLE (MLLE)、Hessian LLE (HLLE)、谱嵌入 (Spectral Embedding)、t-分布随机邻域嵌入 (t-SNE)
\end{itemize}

使用总绝对误差 (TAE)、可信度 (Trustworthiness) 和连续性 (Continuity) 作为评估指标。

\section{文件结构}

项目的文件结构如下:

\begin{lstlisting}[language=bash]
Project/
├── data/
│   └── face.mat         # 包含33张人脸图像的MATLAB数据文件
├── code/
│   ├── main.ipynb       # 主Jupyter笔记本,包含所有代码和可视化
│   ├── utils.py         # 工具函数,包括度量计算和数据处理
│   └── visualization.py # 可视化相关函数
├── results/
│   ├── figures/         # 生成的图表,如散点图、参数调优图等
│   └── metrics/         # 保存的评估指标结果
└── report/
    ├── main.tex         # 报告的主LaTeX文件
    ├── references.bib   # 参考文献BibTeX文件
    └── neurips_2019.sty # 报告使用的样式文件
\end{lstlisting}

\section{代码内容详解}

\subsection{数据加载与预处理}

数据加载部分从MATLAB文件中读取人脸图像数据并进行预处理:

\begin{lstlisting}[language=Python]
import numpy as np
import scipy.io as sio
from sklearn.preprocessing import StandardScaler

# 加载数据
data = sio.loadmat('data/face.mat')
faces = data['images']  # 形状为 [92, 112, 33]

# 将图像展平为向量
n_samples = faces.shape[2]
n_pixels = faces.shape[0] * faces.shape[1]
X = np.zeros((n_samples, n_pixels))
for i in range(n_samples):
    X[i] = faces[:, :, i].reshape(n_pixels)

# 标准化数据
X_scaled = StandardScaler().fit_transform(X)
\end{lstlisting}

\subsection{流形学习方法实现}

使用scikit-learn库实现各种降维方法并进行参数调优:

\begin{lstlisting}[language=Python]
from sklearn.decomposition import PCA
from sklearn.manifold import (
    MDS, Isomap, LocallyLinearEmbedding, 
    SpectralEmbedding, TSNE
)

# 定义要测试的方法
methods = {
    'PCA': PCA(n_components=2),
    'MDS': MDS(n_components=2, metric=True, n_jobs=-1),
    'Isomap': Isomap(n_components=2, n_neighbors=5),
    'LLE': LocallyLinearEmbedding(
        n_components=2, n_neighbors=5, method='standard'),
    'LTSA': LocallyLinearEmbedding(
        n_components=2, n_neighbors=10, method='ltsa'),
    'MLLE': LocallyLinearEmbedding(
        n_components=2, n_neighbors=12, method='modified'),
    'HLLE': LocallyLinearEmbedding(
        n_components=2, n_neighbors=10, method='hessian'),
    'Spectral': SpectralEmbedding(
        n_components=2, n_neighbors=5),
    't-SNE': TSNE(
        n_components=2, perplexity=20, random_state=42)
}

# 使用最优参数对所有方法进行嵌入
embeddings = {}
for name, method in methods.items():
    embeddings[name] = method.fit_transform(X_scaled)
\end{lstlisting}

\subsection{参数调优}

对于基于邻域的方法,针对邻居数量k进行调优,对于t-SNE,针对复杂度参数p进行调优:

\begin{lstlisting}[language=Python]
# 邻居数调优
k_range = [3, 4, 5, 6, 7, 8, 9, 10, 12]
k_methods = ['Isomap', 'LLE', 'LTSA', 'MLLE', 'HLLE', 'Spectral']

k_results = {}
for method_name in k_methods:
    k_results[method_name] = {'TAE': [], 'T': [], 'C': []}
    for k in k_range:
        # HLLE需要k > 5才能进行2D嵌入
        if method_name == 'HLLE' and k <= 5:
            continue
            
        # 创建方法实例并拟合
        if method_name == 'Isomap':
            method = Isomap(n_components=2, n_neighbors=k)
        elif method_name in ['LLE', 'LTSA', 'MLLE', 'HLLE']:
            method_type = {
                'LLE': 'standard', 
                'LTSA': 'ltsa', 
                'MLLE': 'modified', 
                'HLLE': 'hessian'
            }[method_name]
            method = LocallyLinearEmbedding(
                n_components=2, n_neighbors=k, method=method_type)
        elif method_name == 'Spectral':
            method = SpectralEmbedding(n_components=2, n_neighbors=k)
            
        # 计算嵌入和度量
        Y = method.fit_transform(X_scaled)
        tae = compute_tae(Y, ground_truth)
        t, c = compute_trustworthiness_continuity(X_scaled, Y, k=5)
        
        # 存储结果
        k_results[method_name]['TAE'].append(tae)
        k_results[method_name]['T'].append(t)
        k_results[method_name]['C'].append(c)

# t-SNE复杂度调优
p_range = [5, 8, 10, 13, 15, 18, 20, 25, 30]
p_results = {'TAE': [], 'T': [], 'C': []}

for p in p_range:
    tsne = TSNE(n_components=2, perplexity=p, random_state=42)
    Y = tsne.fit_transform(X_scaled)
    
    tae = compute_tae(Y, ground_truth)
    t, c = compute_trustworthiness_continuity(X_scaled, Y, k=5)
    
    p_results['TAE'].append(tae)
    p_results['T'].append(t)
    p_results['C'].append(c)
\end{lstlisting}

\subsection{评估指标计算}

计算用于评估方法性能的三个关键度量:

\begin{lstlisting}[language=Python]
def compute_tae(Y, ground_truth):
    """计算总绝对误差(Total Absolute Error)"""
    # 基于第一个嵌入维度排序
    dim1_order = np.argsort(Y[:, 0])
    
    # 计算与ground truth的绝对差之和
    tae_asc = sum(abs(np.argsort(dim1_order) - np.argsort(ground_truth)))
    
    # 考虑降序情况
    tae_desc = sum(abs(np.argsort(dim1_order[::-1]) - np.argsort(ground_truth)))
    
    # 返回较小值
    return min(tae_asc, tae_desc)

def compute_trustworthiness_continuity(X, Y, k=5):
    """计算可信度(Trustworthiness)和连续性(Continuity)"""
    from sklearn.neighbors import NearestNeighbors
    
    n = X.shape[0]
    
    # 找出在原始空间的k近邻
    nbrs_X = NearestNeighbors(n_neighbors=k+1).fit(X)
    indices_X = nbrs_X.kneighbors(X, return_distance=False)
    # 移除自身
    indices_X = indices_X[:, 1:]
    
    # 找出在嵌入空间的k近邻
    nbrs_Y = NearestNeighbors(n_neighbors=k+1).fit(Y)
    indices_Y = nbrs_Y.kneighbors(Y, return_distance=False)
    # 移除自身
    indices_Y = indices_Y[:, 1:]
    
    # 计算Trustworthiness (T)
    T = 0
    for i in range(n):
        for j in indices_Y[i]:
            if j not in indices_X[i]:
                # 找出j在原始空间k近邻排名
                rank = np.where(np.argsort(np.linalg.norm(X - X[i], axis=1))==j)[0][0]
                T += (rank - k)
    
    T = 1 - 2/(n*k*(2*n - 3*k - 1)) * T
    
    # 计算Continuity (C)
    C = 0
    for i in range(n):
        for j in indices_X[i]:
            if j not in indices_Y[i]:
                # 找出j在嵌入空间k近邻排名
                rank = np.where(np.argsort(np.linalg.norm(Y - Y[i], axis=1))==j)[0][0]
                C += (rank - k)
    
    C = 1 - 2/(n*k*(2*n - 3*k - 1)) * C
    
    return T, C
\end{lstlisting}

\subsection{可视化代码}

结果可视化主要包括散点图和带人脸图像的嵌入结果:

\begin{lstlisting}[language=Python]
import matplotlib.pyplot as plt
import seaborn as sns

# 散点图可视化
def plot_embeddings(embeddings, figsize=(15, 12)):
    n_methods = len(embeddings)
    fig, axes = plt.subplots(3, 3, figsize=figsize)
    axes = axes.flatten()
    
    for i, (name, Y) in enumerate(embeddings.items()):
        ax = axes[i]
        scatter = ax.scatter(Y[:, 0], Y[:, 1], c=range(Y.shape[0]), 
                  cmap='viridis', s=50)
        ax.set_title(name)
        ax.set_xticks([])
        ax.set_yticks([])
    
    plt.tight_layout()
    return fig

# 带人脸图像的嵌入可视化
def plot_embedding_with_faces(Y, faces, title, figsize=(10, 10)):
    fig, ax = plt.subplots(figsize=figsize)
    
    # 散点图
    ax.scatter(Y[:, 0], Y[:, 1], alpha=0)
    
    # 添加小图像
    for i in range(Y.shape[0]):
        face_img = faces[:, :, i]
        # 创建小图像
        img = ax.imshow(face_img, extent=(Y[i, 0]-0.05, Y[i, 0]+0.05, 
                                        Y[i, 1]-0.05, Y[i, 1]+0.05),
                      aspect='auto', cmap='gray', alpha=0.9)
    
    ax.set_title(title)
    ax.set_xticks([])
    ax.set_yticks([])
    
    return fig

# 参数调优可视化
def plot_tuning_results(k_results, k_range, figsize=(15, 10)):
    methods = list(k_results.keys())
    metrics = ['TAE', 'T', 'C']
    
    fig, axes = plt.subplots(len(metrics), 1, figsize=figsize)
    
    for i, metric in enumerate(metrics):
        ax = axes[i]
        for method in methods:
            # 某些方法在特定k值下可能没有结果
            valid_k = []
            valid_vals = []
            
            for j, k in enumerate(k_range):
                if j < len(k_results[method][metric]):
                    valid_k.append(k)
                    valid_vals.append(k_results[method][metric][j])
            
            ax.plot(valid_k, valid_vals, marker='o', label=method)
        
        ax.set_xlabel('Number of neighbors (k)')
        ax.set_ylabel(metric)
        ax.set_xticks(k_range)
        ax.legend()
        
        if metric == 'TAE':
            ax.set_title('Total Absolute Error (lower is better)')
        else:
            ax.set_title(f'{"Trustworthiness" if metric=="T" else "Continuity"} (higher is better)')
    
    plt.tight_layout()
    return fig
\end{lstlisting}

\section{实验结果汇总}

实验结果展示了各种方法的性能比较:

\begin{table}[h]
  \centering
  \caption{各方法的最优参数和评估指标}
  \begin{tabular}{lccc}
    \toprule
    方法 & 最优参数 & TAE (min) & Trustworthiness \& Continuity \\
    \midrule
    Isomap & k=5 & 6 & 高 (>0.98) \\
    LLE & k=5 & 14 & 高 (>0.96) \\
    Spectral & k=5 & 14 & 高 (>0.95) \\
    LTSA & k=10 & 16 & 中高 (>0.93) \\
    HLLE & k=10 & 16 & 中高 (>0.93) \\
    MLLE & k=12 & 22 & 高 (>0.94) \\
    PCA & - & 30 & 高 (>0.97) \\
    t-SNE & p=20 & 90 & 高 (>0.97) \\
    MDS & - & 350 & 高 (>0.98) \\
    \bottomrule
  \end{tabular}
\end{table}

\section{结论}

本项目的主要发现包括:

\begin{enumerate}
    \item Isomap (k=5) 在人脸姿态排序任务中表现最佳,TAE最低。
    \item 非线性流形学习方法(Isomap, LLE, Spectral, HLLE, LTSA)明显优于线性方法(PCA, MDS)和t-SNE。
    \item 对于基于邻域的方法,最优邻居数通常在5-10之间,这表明保持局部结构对于恢复旋转流形至关重要。
    \item 大多数方法都能保持良好的局部结构(高T\&C值),即使全局排序性能(TAE)差异较大。
    \item t-SNE专注于局部结构保持,不适合恢复全局排序关系。
\end{enumerate}

\section{运行环境}

项目使用以下软件环境开发和测试:

\begin{itemize}
    \item Python 3.9
    \item NumPy 1.24.3
    \item SciPy 1.11.3
    \item scikit-learn 1.3.2
    \item Matplotlib 3.8.0
    \item Seaborn 0.12.2
    \item Pandas 2.1.1
\end{itemize}

\end{document} 